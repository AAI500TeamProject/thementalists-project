\section{Conclusions and Recommendations}
\label{sec:con}

This research employed a combination of linear regression, generalized linear models (GLM), and advanced machine learning algorithms, including artificial neural networks, random forests, K-nearest neighbors (KNN), and support vector regression (SVR), to analyze the relationship between mental disorders and the prevalence of eating disorders. By applying correlation matrices, covariance matrices, and regression analysis on standardized data, the study not only identified the relevant factors but also evaluated the predictive power of each model.
The application of a training and testing set separation strategy ensures objectivity and highlights the generalizability of the models when applied to real-world data. The results of the study indicate that mental disorders, especially bipolar disorder and schizophrenia, are strong predictors of the prevalence of eating disorders. Models such as GLM and Random Forest showed high predictive performance and good explanatory power, while neural networks provided reasonable complementary insights. The consistency of statistical indicators, such as correlation, $R^2$, MSE, and confidence intervals, strengthened the confidence in the findings.

In addition, extending the analysis to data on universal health coverage (UHC) and national depression prevalence provides a more comprehensive context for understanding the influence of health system factors on community mental health. Therefore, research contributes not only a way in which further to understand the link between mental disorders and eating disorders but also serves as a framework for applicable public health policy and intervention approaches for the prevention and treatment of eating disorders, eating behaviors, and mental health. This research also serves as a foundation for future studies to develop and deploy practical and widely useful predictive models for mental health care.

While exploring the link between Universal Health Coverage and depression, this analysis concludes that countries with better healthcare systems tend to have lower depression rates (World Health Organization, 2022). However, when focusing on a specific case study comparing the healthcare systems of the United States and Sweden, the findings show that even though the United States has a higher UHC index, it still has a higher rate of depression. These results suggest that additional factors, including the quality of care, access to mental health services, and broader social conditions, play a crucial role in determining national mental health outcomes (Patel et al., 2018).

\subsection{Future work}

Future research should explore how additional factors, such as income, education, and social support, interact with health coverage to impact depression rates (Allen et al., 2014; Patel et al., 2018). It may also be beneficial to analyze policy variations, monitor changes within countries over time, or concentrate on specific subgroups to gain a clearer understanding of which populations derive the most benefit from expanded healthcare (World Health Organization, 2022). Investigating these areas could offer a more comprehensive insight into the connections between health systems and mental health outcomes.
