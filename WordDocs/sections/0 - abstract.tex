Mental illness conditions impact individuals in our society to a large extent. Identification of associated conditions such as eating disorders is crucial for early intervention. This project delves into predictive associations between various psychiatric conditions; we are interested in bipolar disorder, anxiety, schizophrenia, and the incidence of eating disorders, using data from global mental health databases.
An exploratory statistical data analysis was performed by applying univariate, bivariate, and multivariate statistics to uncover patterns and relationships. This study further examines how the quality of care a patient receives at healthcare relates to national depression levels.
Different predictive models were experimented with, that is, generalized linear models, closest neighbor k, random forests, neural networks, and support vector regression. Among these, the Random Forest Regressor was the best with an R² value of 0.995 for the test set. It can be seen from the results that bipolar and schizophrenia disorders are the best predictors of eating disorders. In addition, the Swedish-United States case study highlighted the role of broader health system characteristics on outcomes in mental health. The study provides an evidence-based model for identifying at-risk groups for eating disorders and informs public health policy with the objective of improving outcomes in mental health.